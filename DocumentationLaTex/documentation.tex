\documentclass[a4paper,12pt,abstracton,titlepage]{scrartcl}

\usepackage[ngerman]{babel}
\usepackage[T1]{fontenc}
\usepackage[utf8]{inputenc} % Umlaute, evtl. vom Betriebssystem abhaengig
\usepackage{lmodern}
\usepackage{pgfgantt}
\usepackage{titlesec}
\usepackage{float}
\usepackage{floatflt}
\usepackage{blindtext}
\usepackage{amsmath}
\usepackage{tabularx,url}


\titleformat*{\section}{\large\bfseries}
\titleformat*{\subsection}{\large\bfseries}
\titleformat*{\subsubsection}{\large\bfseries}
\titleformat*{\paragraph}{\large\bfseries}
\titleformat*{\subparagraph}{\large\bfseries}

\renewcaptionname{ngerman}{\figurename}{Abb.}

%\titlehead{Ulm University}
%\title{Title}
%\subject{Subject}
%\author{Author}
%\publishers{%
	%\rule{\textwidth}{0.4pt} \\
	%\vspace{0.5cm}
    %\normalfont\normalsize%
    %\parbox{0.9\linewidth}{%
    %    Abstract or Introduction
    %} \\
    %\vspace{0.5cm}
   	%\rule{\textwidth}{0.4pt}
%}

\begin{document}
%\maketitle

%%% begin costom title
{\Large\noindent \emph{ESIEE Paris}}

{\Large\noindent \emph{IGI-3008}}
\begin{center}
	{\large Cryptographie	 \\ \large Léa MENERET, Fathima SAHADATTALY, Ulrike KULZER \\ \today}
\end{center}
%%% end custom title

\setcounter{page}{1} % reset page counter to one for the first page, leave the title page out

\section{Contexte}
\subsection{En général}
La cryptographie est une technique utilisée pour rendre incompréhensible à autrui un message entre un expéditeur et un destinataire. Ce procédé a notamment été utilisé en période de guerre pour permettre des attaques surprises. 
Le principe est le suivant : L’expéditeur à partir d’une clé crypte son message et l’envoie au destinataire. Celui-ci possède aussi la clé qui va lui permettre ainsi de décrypter le message.

\subsection{Histoire}
La cryptographie est utilisée depuis l’Antiquité mais certaines de ces méthodes les plus abouties datent du 20e siècle. Il existe différents principes de cryptage plus ou moins compliqués tels que


\section{Fonctionnalité}
\subsection{En général}
Lorem ipsum

\subsection{En détail et coupé en modules}


\section{Interfaces utilisateurs}





% (CIE\footnote{CIE = Commission internationale de l’éclairage})  "`Inviwo"'\footnote{Interactive Visualization Workshop}

%\section{Grundlagen}
\begin{floatingfigure}[r]{6cm}
%	\includegraphics[width=0.3\textwidth]{./Bilder/1cie.png}
	\caption{CIE"=Normvalenzsystem}
	\label{cie}
\end{floatingfigure}

%(meist im CIE"=Normvalenzsystem) oder verschiedenen Farbkörpern, z. B. Würfel oder Kegeln, dargestellt. Als Beispiel sieht man in Abbildung \ref{cie} eine Darstellung des Standard"=RGB"=Farbraums


\begin{floatingfigure}[l]{5.3cm}
%	\includegraphics[width=0.25\textwidth]{./Bilder/2venn.png}
	\caption{Prinzip der additiven Farbmischung}
	\label{venn}
\end{floatingfigure}

\par

% wodurch weißes Licht reflektiert wird (s. Abb. \ref{venn}). 


\vspace{2em}
%\section{Farbmodell, Konvertierung und Implementierung}
%\subsection{Vorstellung der Farbmodelle}
%\subsubsection{Das Farbmodell RGB} 
%(für Genaueres siehe \cite{kompendium}).

\subsubsection{Das Farbmodell HSV}
\begin{floatingfigure}[r]{8.3cm}
	%\includegraphics[width=0.5\textwidth]{./Bilder/hsvKegel.png}
	\caption{HSV-Farbkörper}
	\label{hsv}
\end{floatingfigure} 


\vspace{1em}
\subsubsection{Die Farbmodelle YUV, YPbPr und YCbCr}
% NTSC\footnote{für Details siehe \cite{linkfangNTSC}} und des PAL-Verfahrens\footnote{für Details siehe \cite{linkfangPAL}} entwickelt und basiert auf dem RGB"= Farbmodell. 

%Von diesen bestimmt man für die weiteren Rechnungen das Maximum max\_value und Minimum min\_value mit der entsprechenden Funktion.
\[    hue := \left\{\begin{array}{lr}
        0.0, & \text{for }\; max\_value = min\_value\\
        60.0 \cdot \left( 0.0 + \frac {g - b} {max\_value - min\_value} \right), & \text{for }\; max\_value = r\\
        60.0 \cdot \left( 2.0 + \frac {b - r} {max\_value - min\_value} \right), & \text{for }\; max\_value = g\\
        60.0 \cdot \left( 4.0 + \frac {r - g} {max\_value - min\_value} \right), & \text{for }\; max\_value = b
        \end{array}\right.
  \]  


\[
    satVal := \left\{\begin{array}{lr}
        0.0, & \text{for }\; max\_value = 0.0\\
        \frac {max\_value - min\_value} {max\_value}), & \text{otherwise }\; \\
        \end{array}\right.
  \]  


\begin{figure}[htbp]
\begin{minipage}[t]{0.48\textwidth}
  \begin{center}
    %\includegraphics[height=5.6cm]{./Bilder/bilderKonv/original/original.jpg}
    \caption{Originalbild}
    \label{originalScot}
  \end{center}
\end{minipage}
\begin{minipage}[t]{0.52\textwidth}
  \begin{center}
    %\includegraphics[height=5.6cm]{./Bilder/bilderKonv/original/canvas_rgb2hsv_cut.png}
    \caption{Bild nach der Konvertierung}
    \label{rgb2hsv}
  \end{center}
\end{minipage}
\end{figure}


%e floor()"=Funktion verwendet, die äquivalent zu den Gaußklammern nach unten ist. Zusätzlich zu i werden vier weitere Hilfsvariablen definiert: \[f := hue - i,\qquad p := val \cdot (1.0 - sat),\qquad q := val \cdot (1.0 - (sat \cdot f))\] und \[t := val \cdot (1.0 - sat \cdot (1.0 - f))\].
% woher kommen diese Hilfsvariablen??
% nach Gaußklammern: \[i := \lfloor hue \rfloor\]


\[
  (r,g,b)^T := \left\{\begin{array}{lr}
        (val,t,p)^T, & \text{for }\; i = 0\\
        (q,val,p)^T, & \text{for }\; i = 1\\
        (p,val,t)^T, & \text{for }\; i = 2\\
        (p,q,val)^T, & \text{for }\; i = 3\\
        (t,p,val)^T, & \text{for }\; i = 4\\
        (val,p,q)^T, & \text{for }\; i = 5\\
        \end{array}\right.
  \]
  

\begin{figure}[htbp]
\begin{minipage}[t]{0.48\textwidth}
  \begin{center}
   % \includegraphics[height=5.6cm]{./Bilder/bilderKonv/original/canvas_rgb2hsv_cut.png}
    \caption{Bild im HSV"=Farbraum}
    \label{hsvVorher}
  \end{center}
\end{minipage}
\begin{minipage}[t]{0.52\textwidth}
  \begin{center}
   % \includegraphics[height=5.6cm]{./Bilder/bilderKonv/original/canvas_hsv2rgb_cut.png}
    \caption{Bild nach der Konvertierung}
    \label{hsv2rgb}
  \end{center}
\end{minipage}
\end{figure}

% "`Standard Definition Television"', HDTV für "`High Definition Television'".

\[
    \left(\begin{array}{c} y \\ cb \\ cr \end{array}\right) =
    \left(\begin{array}{c} 0 \\ 128 \\ 128 \end{array}\right)
    +
    \begin{pmatrix} 0.299 & 0.587 & 0.114 \\ -0.169 & -0.331 & 0.500 \\ 0.500 & -0.419 & -0.081
    \end{pmatrix} \cdot \left(\begin{array}{c} r \\ g \\ b \end{array}\right)
\]


\begin{figure}[htbp]
\begin{minipage}[t]{0.48\textwidth}
  \begin{center}
    %\includegraphics[height=5.6cm]{./Bilder/bilderKonv/original/original.jpg}
    \caption{Originalbild}
    \label{originalY}
  \end{center}
\end{minipage}
\begin{minipage}[t]{0.52\textwidth}
  \begin{center}
    %\includegraphics[height=5.6cm]{./Bilder/bilderKonv/original/canvas_rgb2ycbcr_cut.png}
    \caption{Bild nach der Konvertierung}
    \label{rgb2ycbcr}
  \end{center}
\end{minipage}
\end{figure}
Zur Übersicht sind in den Abbildungen 14 bis 16 die einzelnen Farbkomponenten getrennt dargestellt (s. Abb. \ref{Y}, Abb. \ref{Cb}, Abb. \ref{Cr}).


\begin{figure}[htbp]
\begin{minipage}[t]{0.31\textwidth}
  \begin{center}
    %\includegraphics[height=3.7cm]{./Bilder/bilderKonv/original/canvas_y_cut.png}
    \caption{nur Y}
    \label{Y}
  \end{center}
\end{minipage}
\begin{minipage}[t]{0.31\textwidth}
  \begin{center}
    %\includegraphics[height=3.7cm]{./Bilder/bilderKonv/original/canvas_cb_cut.png}
    \caption{nur Cb}
    \label{Cb}
  \end{center}
\end{minipage}
\begin{minipage}[t]{0.31\textwidth}
  \begin{center}
    %\includegraphics[height=3.7cm]{./Bilder/bilderKonv/original/canvas_cr_cut.png}
    \caption{nur Cr}
    \label{Cr}
  \end{center}
\end{minipage}
\end{figure}

 
\[
    \left(\begin{array}{c} r \\ g \\ b \end{array}\right) =
    \begin{pmatrix} 1.000 & 1.000 & 1.000 \\ 0.000 & -0.343 & -0.419 \\ 1.400 & -0.711 & 0.000
    
    \end{pmatrix}
    \cdot \left(\begin{array}{c} y \\ cb - 128 \\ cr - 128 \end{array}\right)
\]

 


\begin{figure}[htbp]
\begin{minipage}[t]{0.48\textwidth}
  \begin{center}
    %\includegraphics[height=5.6cm]{./Bilder/bilderKonv/original/canvas_rgb2ycbcr_cut.png}
    \caption{Bild im YCbCr"=Farbraum}
    \label{ycbcr}
  \end{center}
\end{minipage}
\begin{minipage}[t]{0.52\textwidth}
  \begin{center}
    %\includegraphics[height=5.6cm]{./Bilder/bilderKonv/original/canvas_ycbcr2rgb_cut.png}
    \caption{Bild nach der Konvertierung}
    \label{ycbcr2rgb}
  \end{center}
\end{minipage}
\end{figure}


\newpage
\begin{figure}[htbp]
\begin{minipage}[t]{0.48\textwidth}
  \begin{center}
    %\includegraphics[height=5cm]{./Bilder/bilderKonv/balloons_fehlerhaft/rgb2hsv_cut.png}
    \caption{RGB nach HSV}
    \label{errorBall}
    \begin{small}
    Quelle: w.hsv
    \end{small}
  \end{center}
\end{minipage}
\begin{minipage}[t]{0.48\textwidth}
  \begin{center}
    %\includegraphics[height=5cm]{./Bilder/bilderKonv/balloons_fehlerhaft/testGes.png}
    \caption{HSV nach RGB mit vergrößertem Ausschnitt}
    \label{errorBallRGB}
  \end{center}
\end{minipage}
\end{figure}



\section{Referenzen}
\nocite{*} 
%\renewcommand{\bibname}{\section{Sources}} % Redefine bibname
\bibliographystyle{IEEEtran} % Set any options you want
\bibliography{bibliography} % Build the bibliography

\subsection*{Bildquellen}

\begin{tabularx}{0.95\linewidth}{@{}>{\bfseries}l@{\hspace{0.5em}}X@{}}
    Abb. \ref{cie}   &     \url{http://www.itwissen.info/definition/lexikon/Standard-RGB-sRGB-standard-RGB.html} (31.01.2017)
    \\
    Abb. \ref{venn}   &	   \url{https://www.saxoprint.de/blog/der-farbraum-rgb-und-cmyk-im-vergleich/} (31.01.2017)
    \\
    Abb. \ref{hsv}   &     \url{http://de.mathworks.com/help/images/convert-from-hsv-to-rgb-color-space.html?requestedDomain=www.mathworks.com} (07.02.2017)
    \\
    Abb. \ref{originalScot}   &     Ulrike Kulzer
    \\
    Abb. \ref{errorBall}   &     Originalbild:
    \url{https://www.tutorialspoint.com/java_dip/color_space_conversion.htm} (08.02.2017)
    \\
\end{tabularx}





\end{document}




% LABEL UNTER CAPTION

